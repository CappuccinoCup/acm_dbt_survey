%%
%% This is file `sample-acmsmall.tex',
%% generated with the docstrip utility.
%%
%% The original source files were:
%%
%% samples.dtx  (with options: `acmsmall')
%% 
%% IMPORTANT NOTICE:
%% 
%% For the copyright see the source file.
%% 
%% Any modified versions of this file must be renamed
%% with new filenames distinct from sample-acmsmall.tex.
%% 
%% For distribution of the original source see the terms
%% for copying and modification in the file samples.dtx.
%% 
%% This generated file may be distributed as long as the
%% original source files, as listed above, are part of the
%% same distribution. (The sources need not necessarily be
%% in the same archive or directory.)
%%
%%
%% Commands for TeXCount
%TC:macro \cite [option:text,text]
%TC:macro \citep [option:text,text]
%TC:macro \citet [option:text,text]
%TC:envir table 0 1
%TC:envir table* 0 1
%TC:envir tabular [ignore] word
%TC:envir displaymath 0 word
%TC:envir math 0 word
%TC:envir comment 0 0
%%
%%
%% The first command in your LaTeX source must be the \documentclass
%% command.
%%
%% For submission and review of your manuscript please change the
%% command to \documentclass[manuscript, screen, review]{acmart}.
%%
%% When submitting camera ready or to TAPS, please change the command
%% to \documentclass[sigconf]{acmart} or whichever template is required
%% for your publication.
%%
%%
\documentclass[acmsmall]{acmart}
\settopmatter{printccs=false, printacmref=false}
\renewcommand\footnotetextcopyrightpermission[1]{}

\usepackage{makecell}

%%
%% \BibTeX command to typeset BibTeX logo in the docs
\AtBeginDocument{%
  \providecommand\BibTeX{{%
    Bib\TeX}}}

%% Rights management information.  This information is sent to you
%% when you complete the rights form.  These commands have SAMPLE
%% values in them; it is your responsibility as an author to replace
%% the commands and values with those provided to you when you
%% complete the rights form.
% \setcopyright{acmcopyright}
% \copyrightyear{2018}
% \acmYear{2018}
% \acmDOI{XXXXXXX.XXXXXXX}
\setcopyright{none}

%%
%% These commands are for a JOURNAL article.
\acmJournal{JACM}
\acmVolume{1}
\acmNumber{1}
\acmArticle{1}
\acmMonth{1}

%%
%% Submission ID.
%% Use this when submitting an article to a sponsored event. You'll
%% receive a unique submission ID from the organizers
%% of the event, and this ID should be used as the parameter to this command.
%%\acmSubmissionID{123-A56-BU3}

%%
%% For managing citations, it is recommended to use bibliography
%% files in BibTeX format.
%%
%% You can then either use BibTeX with the ACM-Reference-Format style,
%% or BibLaTeX with the acmnumeric or acmauthoryear sytles, that include
%% support for advanced citation of software artefact from the
%% biblatex-software package, also separately available on CTAN.
%%
%% Look at the sample-*-biblatex.tex files for templates showcasing
%% the biblatex styles.
%%

%%
%% The majority of ACM publications use numbered citations and
%% references.  The command \citestyle{authoryear} switches to the
%% "author year" style.
%%
%% If you are preparing content for an event
%% sponsored by ACM SIGGRAPH, you must use the "author year" style of
%% citations and references.
%% Uncommenting
%% the next command will enable that style.
%%\citestyle{acmauthoryear}

% inlined bib file
\usepackage{filecontents}

%-------------------------------------------------------------------------------
\begin{filecontents}{\jobname.bib}
%-------------------------------------------------------------------------------

@inproceedings{DBLP:conf/icsm/CifuentesM96,
  author       = {Cristina Cifuentes and
                  Vishv M. Malhotra},
  title        = {Binary Translation: Static, Dynamic, Retargetable?},
  booktitle    = {1996 International Conference on Software Maintenance {(ICSM} '96),
                  4-8 November 1996, Monterey, CA, USA, Proceedings},
  pages        = {340--349},
  publisher    = {{IEEE} Computer Society},
  year         = {1996},
  url          = {https://doi.org/10.1109/ICSM.1996.565037},
  doi          = {10.1109/ICSM.1996.565037},
  timestamp    = {Fri, 24 Mar 2023 00:04:11 +0100},
  biburl       = {https://dblp.org/rec/conf/icsm/CifuentesM96.bib},
  bibsource    = {dblp computer science bibliography, https://dblp.org}
}

@article{DBLP:journals/computer/AltmanKS00,
  author       = {Erik R. Altman and
                  David R. Kaeli and
                  Yaron Sheffer},
  title        = {Welcome to the Opportunities of Binary Translation},
  journal      = {Computer},
  volume       = {33},
  number       = {3},
  pages        = {40--45},
  year         = {2000},
  url          = {https://doi.org/10.1109/2.825694},
  doi          = {10.1109/2.825694},
  timestamp    = {Mon, 26 Oct 2020 08:21:49 +0100},
  biburl       = {https://dblp.org/rec/journals/computer/AltmanKS00.bib},
  bibsource    = {dblp computer science bibliography, https://dblp.org}
}

@article{DBLP:journals/pieee/AltmanEGS01,
  author       = {Erik R. Altman and
                  Kemal Ebcioglu and
                  Michael Gschwind and
                  Sumedh Sathaye},
  title        = {Advances and future challenges in binary translation and optimization},
  journal      = {Proc. {IEEE}},
  volume       = {89},
  number       = {11},
  pages        = {1710--1722},
  year         = {2001},
  url          = {https://doi.org/10.1109/5.964447},
  doi          = {10.1109/5.964447},
  timestamp    = {Tue, 17 Aug 2021 21:07:56 +0200},
  biburl       = {https://dblp.org/rec/journals/pieee/AltmanEGS01.bib},
  bibsource    = {dblp computer science bibliography, https://dblp.org}
}

@inproceedings{probst2002dynamic,
  title={Dynamic binary translation},
  author={Probst, Mark},
  booktitle={UKUUG Linux Developer’s Conference},
  volume={2002},
  year={2002}
}

@article{DBLP:journals/pieee/Duesterwald05,
  author       = {Evelyn Duesterwald},
  title        = {Design and Engineering of a Dynamic Binary Optimizer},
  journal      = {Proc. {IEEE}},
  volume       = {93},
  number       = {2},
  pages        = {436--448},
  year         = {2005},
  url          = {https://doi.org/10.1109/JPROC.2004.840302},
  doi          = {10.1109/JPROC.2004.840302},
  timestamp    = {Fri, 02 Oct 2020 14:42:17 +0200},
  biburl       = {https://dblp.org/rec/journals/pieee/Duesterwald05.bib},
  bibsource    = {dblp computer science bibliography, https://dblp.org}
}

@article{Li2007,
  title={Dynamic Binary Translation and Optimization},
  author={Jianhui Li and Xiangning Ma and Chuanqi Zhu},
  journal={Journal of Computer Research and Development},
  volume={44},
  number={1},
  pages={161--168},
  year={2007}
}

@article{DBLP:journals/csur/WenzlMUW19,
  author       = {Matthias Wenzl and
                  Georg Merzdovnik and
                  Johanna Ullrich and
                  Edgar R. Weippl},
  title        = {From Hack to Elaborate Technique - {A} Survey on Binary Rewriting},
  journal      = {{ACM} Comput. Surv.},
  volume       = {52},
  number       = {3},
  pages        = {49:1--49:37},
  year         = {2019},
  url          = {https://doi.org/10.1145/3316415},
  doi          = {10.1145/3316415},
  timestamp    = {Sat, 08 Jan 2022 02:23:10 +0100},
  biburl       = {https://dblp.org/rec/journals/csur/WenzlMUW19.bib},
  bibsource    = {dblp computer science bibliography, https://dblp.org}
}

@inproceedings{DBLP:conf/usenix/Bellard05,
  author       = {Fabrice Bellard},
  title        = {QEMU, a Fast and Portable Dynamic Translator},
  booktitle    = {Proceedings of the {FREENIX} Track: 2005 {USENIX} Annual Technical
                  Conference, April 10-15, 2005, Anaheim, CA, {USA}},
  pages        = {41--46},
  publisher    = {{USENIX}},
  year         = {2005},
  url          = {http://www.usenix.org/events/usenix05/tech/freenix/bellard.html},
  timestamp    = {Mon, 01 Feb 2021 08:43:55 +0100},
  biburl       = {https://dblp.org/rec/conf/usenix/Bellard05.bib},
  bibsource    = {dblp computer science bibliography, https://dblp.org}
}

\end{filecontents}

%%
%% end of the preamble, start of the body of the document source.
\begin{document}

%%
%% The "title" command has an optional parameter,
%% allowing the author to define a "short title" to be used in page headers.
\title{A Survey of Dynamic Binary Translation and Optimization}

%%
%% The "author" command and its associated commands are used to define
%% the authors and their affiliations.
%% Of note is the shared affiliation of the first two authors, and the
%% "authornote" and "authornotemark" commands
%% used to denote shared contribution to the research.
\author{Zixing Bai}
\affiliation{%
 \institution{Fudan University}
 \streetaddress{220 Handan Road}
 \city{Shanghai}
 \country{China}}

\author{Qingcan Kong}
\affiliation{%
 \institution{Fudan University}
 \streetaddress{220 Handan Road}
 \city{Shanghai}
 \country{China}}

\author{Hanzhang Wang}
\affiliation{%
 \institution{Fudan University}
 \streetaddress{220 Handan Road}
 \city{Shanghai}
 \country{China}}

\author{Chaoyi Liang}
\affiliation{%
 \institution{Fudan University}
 \streetaddress{220 Handan Road}
 \city{Shanghai}
 \country{China}}

% \author{Lars Th{\o}rv{\"a}ld}
% \affiliation{%
%   \institution{The Th{\o}rv{\"a}ld Group}
%   \streetaddress{1 Th{\o}rv{\"a}ld Circle}
%   \city{Hekla}
%   \country{Iceland}}
% \email{larst@affiliation.org}

% \author{Valerie B\'eranger}
% \affiliation{%
%   \institution{Inria Paris-Rocquencourt}
%   \city{Rocquencourt}
%   \country{France}
% }

% \author{Aparna Patel}
% \affiliation{%
%  \institution{Rajiv Gandhi University}
%  \streetaddress{Rono-Hills}
%  \city{Doimukh}
%  \state{Arunachal Pradesh}
%  \country{India}}

% \author{Huifen Chan}
% \affiliation{%
%   \institution{Tsinghua University}
%   \streetaddress{30 Shuangqing Rd}
%   \city{Haidian Qu}
%   \state{Beijing Shi}
%   \country{China}}

% \author{Charles Palmer}
% \affiliation{%
%   \institution{Palmer Research Laboratories}
%   \streetaddress{8600 Datapoint Drive}
%   \city{San Antonio}
%   \state{Texas}
%   \country{USA}
%   \postcode{78229}}
% \email{cpalmer@prl.com}

% \author{John Smith}
% \affiliation{%
%   \institution{The Th{\o}rv{\"a}ld Group}
%   \streetaddress{1 Th{\o}rv{\"a}ld Circle}
%   \city{Hekla}
%   \country{Iceland}}
% \email{jsmith@affiliation.org}

% \author{Julius P. Kumquat}
% \affiliation{%
%   \institution{The Kumquat Consortium}
%   \city{New York}
%   \country{USA}}
% \email{jpkumquat@consortium.net}

%%
%% By default, the full list of authors will be used in the page
%% headers. Often, this list is too long, and will overlap
%% other information printed in the page headers. This command allows
%% the author to define a more concise list
%% of authors' names for this purpose.
\renewcommand{\shortauthors}{Bai et al.}

%%
%% The abstract is a short summary of the work to be presented in the
%% article.
\begin{abstract}
  Binary translation is a technology to translate executable binary files from a source Instruction-Set Architecture (ISA) to a target ISA.
  Static Binary Translation (SBT) takes source binaries as input and target binaries as output, while Dynamic Binary Translation (DBT) generates and executes target binaries on-the-fly.
  In the past three decades, DBT has attracted extensive attention and research, and has been widely applied in binary migration, binary analysis, binary optimizations, and simulators.
  In this article, we survey DBT, including its development history, optimizations and applications.
  We compare the difference between DBT and SBT and explain why DBT is more popular.
  We categorize and elaborate on DBT optimizations, as well as various applications.
  Then, we discuss the still unsolved problems in DBT and where the challenges lie, and put forward our own thinking.
\end{abstract}

%%
%% The code below is generated by the tool at http://dl.acm.org/ccs.cfm.
%% Please copy and paste the code instead of the example below.
%%
% \begin{CCSXML}
% <ccs2012>
%  <concept>
%   <concept_id>10010520.10010553.10010562</concept_id>
%   <concept_desc>Computer systems organization~Embedded systems</concept_desc>
%   <concept_significance>500</concept_significance>
%  </concept>
%  <concept>
%   <concept_id>10010520.10010575.10010755</concept_id>
%   <concept_desc>Computer systems organization~Redundancy</concept_desc>
%   <concept_significance>300</concept_significance>
%  </concept>
%  <concept>
%   <concept_id>10010520.10010553.10010554</concept_id>
%   <concept_desc>Computer systems organization~Robotics</concept_desc>
%   <concept_significance>100</concept_significance>
%  </concept>
%  <concept>
%   <concept_id>10003033.10003083.10003095</concept_id>
%   <concept_desc>Networks~Network reliability</concept_desc>
%   <concept_significance>100</concept_significance>
%  </concept>
% </ccs2012>
% \end{CCSXML}

% \ccsdesc[500]{Computer systems organization~Embedded systems}
% \ccsdesc[300]{Computer systems organization~Redundancy}
% \ccsdesc{Computer systems organization~Robotics}
% \ccsdesc[100]{Networks~Network reliability}

%%
%% Keywords. The author(s) should pick words that accurately describe
%% the work being presented. Separate the keywords with commas.
\keywords{dynamic binary translation, dynamic binary instrumentation, static binary translation, code generation}

% \received{20 February 2007}
% \received[revised]{12 March 2009}
% \received[accepted]{5 June 2009}

%%
%% This command processes the author and affiliation and title
%% information and builds the first part of the formatted document.
\maketitle

\section{Introduction}

In the earlier years, the hardware was more diverse, and many different Instruction Set Architectures (ISA) emerged.
With the updating and upgrading of hardwares and ISAs, binary compatibility and migration issues have become problems faced by many binary programs that lack source codes.
Binary translation can deal with the problem by converting binary codes of a guest ISA directly to binary codes of a host ISA without source codes.
Moreover, since binary translation can make full use of the hardware features of the guest and the host machine, and conduct in-depth analysis on binary programs, it has been widely used in code optimization, binary analysis, binary security, virtualization, simulation, and many other fields. 

Binary translation can be divided into Static Binary Translation (SBT) and Dynamic Binary Translation (DBT).
SBT aims to translate all of the binary codes of an executable file into binary codes that can runs in the current host machine without having to running the codes first.
DBT translates and executes binary codes on-the-fly, thus can collect program run-time information and has more powerful translation capabilities.
Since DBT can cover more scenarios, it has received more attention and research, and is also the focus of this survey.

Most of the influential review articles on binary translation we could find so far were written before 2010.
To the best of our knowledge, ~\cite{DBLP:conf/icsm/CifuentesM96} was the first to review the binary translation and discuss the differences between SBT and DBT.
Around 2000, Altman et al.~\cite{DBLP:journals/pieee/AltmanEGS01}\cite{DBLP:journals/computer/AltmanKS00} presented the challenges and opportunities faced by binary translation at the time.
In 2002, Probst ~\cite{probst2002dynamic} presented an overview of DBT.
In 2005, Duesterwald~\cite{DBLP:journals/pieee/Duesterwald05} introduced the design and implementation details of DBT, which is a fundamental part of DBT.
In 2007, Li Jianhui et al. ~\cite{Li2007} analyzed and summarized the research details of DBT.
In 2019, Wenzl et al. ~\cite{DBLP:journals/csur/WenzlMUW19} briefly introduced the related research on DBT.
In the past two decades, researchers have continuously proposed new DBT methods, or applied DBT to simulation, security analysis and other fields.
This article aims to categorize the optimization methods in the field of DBT, and sort out the practical applications of DBT technology.

The Rest of this paper is organized as follows.
In Section~\ref{sec:process}, we present the survey process of this research.
In Section~\ref{sec:background}, we introduces the basic principles and theoretical innovations of binary translation, especially DBT and the advantages of DBT over SBT.
In Section~\ref{sec:optimizations}, we focus on the optimization of DBT, and classify and elaborate various optimization methods.
In Section~\ref{sec:applications}, we introduce the applications of DBT.
Section~\ref{sec:comments} is the characteristic of this survey.
In Section~\ref{sec:comments}, we discuss the unresolved issues and possible problems in the development of DBT.
To address these challenges, we propose some possible solutions.
Then, we comprehensively considered the relationship between DBT and SBT, the historical development of DBT and the research hotspots of DBT in recent years, and put forward the future directions of DBT.
In Section~\ref{sec:conclusions}, we summarize the content of this survey.


\section{Survey Process}


\section{Background}
\label{sec:background}

Binary translation translates binary programs from one ISA to another without relying on source code. 
Binary translation can be divided into static binary translation (SBT), dynamic binary translation (DBT) and binary translation combining static and dynamic. 
These three translation methods will be introduced separately below.

\subsection{Static Binary Translation}
The process of static binary translation is basically similar to compilation, the difference is that the input file of static binary translation changes from source code to binary program: 
the input binary program is first parsed through the disassembly process, and then the control flow graph is generated through analysis, and then further analyzed to generate The intermediate representation is then optimized on the intermediate representation, and finally the binary instructions that can be executed on the host machine are generated through the code generation process.
Since the translation process is performed offline, its execution efficiency is relatively high, and translation can be performed multiple times at a time.
Static binary translation can be regarded as a compiler that uses binary programs as input, so some people propose that the compiler can be directly transformed into a static binary translation tool ~\cite{Hwang2010DisIRerCA}: 
just convert binary instructions into intermediate representations, from intermediate representations to The process of the host computer instruction can be completed directly by using the compiler. 
However, whether it is a compiler or a static binary translator, it is usually optimized on the intermediate representation, so some host information will be lost and optimization opportunities will be missed. 
Therefore, the research of Tan Jie et al. ~\cite{TanJie2017ISSN} focuses on the optimization of the host. 
In terms of code, combined with liveness analysis and peephole optimization theory, a large number of redundant instructions have been successfully eliminated, and the execution efficiency has been improved by a maximum of 42\%.

However, static binary translation cannot well solve the problems faced in indirect jumps, self-modifying code, and dynamic linking.

Indirect jumps are jump instructions that store the jump target in a register instead of an instruction, and these jump targets often have to wait until runtime to determine the specific value.
Determining the indirect jump targets and their mapping on the host ISA is known as code discovery and code localization problems, respectively. 
For code discovery and code localization, researchers have proposed some feasible solutions. 
One solution for code discovery is to establish an address mapping table, store possible jump targets in the table, and insert a mark when translating indirect jump instructions, and when the mark is encountered during execution, it will go to the address mapping table to query the jump Forward address.
Bansal et al. ~\cite{bansal2008binary} use the method of the above address mapping table to determine the indirect jump target.
Chen Long et al. ~\cite{chenlong2008} proposed a semantic map-based method to identify those indirect jump instructions and confirm their jump targets.
After confirming the jump target, the customer target address must be converted to the host target address, that is, code positioning.
Wang Jun et al. \cite{wangjun2019} proposed a solution to establish a secondary mapping table to save the mapping relationship between the client's target address and the host's target address to determine the mapped target address.

However, no effective static solution has been found so far for the problems posed by self-modifying code and dynamic linking.
Due to the mixed characteristics of data and instructions in the von Neumann architecture, for a memory access instruction, it is difficult for static binary translation to distinguish whether the content accessed by the memory access instruction is an instruction or ordinary data during the offline translation process.
Therefore, self-modifying code is hardly recognized by static binary translation, and it is even more difficult to process self-modifying code offline.
However, due to the low proportion of self-modifying codes in common applications, current static binary translation tools tend to avoid them.
Secondly, static binary translation is also difficult to deal with the problems caused by dynamic link libraries.
When a dynamic link library version upgrade occurs at runtime, static binary translation can hardly recognize the occurrence of this situation, unless all dynamic link libraries have prepared translated versions in advance, which is related to the dynamic link library Contrary to the design concept.

\subsection{Dynamic Binary Translation}
Dynamic binary translation puts the translation process at runtime and executes it while translating. 
Although this inevitably brings a lot of translation and optimization overhead at runtime, because of the runtime information, dynamic binary translation can easily solve code discovery and code location problems. 
As for the problem of self-modifying code, there are two mature solutions for dynamic binary translation ~\cite{probst2002dynamic}: special instructions provided by ISA to notify the processor that self-modifying has occurred; or using page protection mechanism to identify and process self-modifying code. 
QEMU~\cite{DBLP:conf/usenix/Bellard05}, which is well known in the industry, uses a page protection mechanism to solve the problem of self-modifying code.

A typical dynamic binary translation tool is mainly composed of translation engine, execution engine and code cache. 
The dynamic binary translation tool takes the binary program file of the client computer ISA as input, and performs translation in units of basic blocks according to certain rules. 
When the basic block is executed for the first time, the translation engine first translates it through four steps of parsing, analysis, optimization, and code generation, and then stores the translated code segment in the code cache and sends it to the execution engine for execution. 
When the basic block is subsequently executed again, the execution engine will take the code segment out of the code cache and send it directly for execution.


\section{Optimizations of DBT}
\label{sec:optimizations}

The main optimization goals of DBT focus on improving translation efficiency and translation quality.
The optimization methods proposed by the researchers can be divided into five optimization categories: parallel optimization, memory optimization, basic block-level optimization, instruction-level optimization, and software-hardware coordination optimization.

\subsection{Parallel Optimization}
Due to the development of multi-core architectures in recent years, some research content has focused on using the concurrency characteristics of multi-core architectures to improve DBT performance.
On the one hand, the translation process can be parallelized [61][53]; And the execution process is placed on a different core to accelerate [62][63][64].
For example, HQEMU[78][79] extends QEMU's translator.
For a multi-threaded program running on QEMU, HQEMU will first add the program to be optimized into a first-in-first-out optimization queue.
The corresponding code is translated and optimized, which reduces the translation overhead.
At the same time, the generated code can also be executed in parallel, which improves the code execution efficiency.
This method is to accelerate DBT with the help of multi-core features by placing translation threads and execution threads on multiple cores.
In addition, the researchers also translated multi-threaded programs [65], special register mapping [27] [28] [29], atomic instructions [30] [31], branch instructions [32] [33], floating point The translation and optimization of instructions [34], and the optimization of helper functions [46] are deeply studied.

\subsection{Memory Optimization}
System-level DBT needs to simulate the entire hardware platform including memory.
Since memory is one of the main bottlenecks that limit the efficiency of program operation, the translation and optimization of memory has become a major research focus of system-level DBT.
Memory-related research mainly focuses on the following aspects: translation and optimization of memory instructions [66], research on data alignment issues [67][68], optimization of memory overhead of DBT itself [69], and hardware transaction hardware transactional memory (HTM) virtualization [71], research on memory address mapping [70] [72], etc.
It is worth mentioning that regarding memory address mapping, since the pure software mapping method is too inefficient, you can introduce a shadow page table and use the host's memory management unit (memory management unit, MMU) [70], or directly design a special Methods such as the MMU [72] for binary translation use hardware to accelerate memory address mapping, or directly map the transactions and operations of the client MMU to the host's memory virtualization mechanism [86].

\subsection{Code Layout Optimization}
As mentioned above, DBT usually translates in units of basic blocks, but the basic block method has a disadvantage that its granularity is too small to perform some in-depth optimization.
Therefore, researchers propose that multiple basic blocks can be combined into one translation unit (region) for overall translation and deep optimization.
One method of constructing a region is to analyze the hot path of program execution by recording the execution information of the program, and then construct all the basic blocks on the hot path into a trace.
Regarding trace, Duesterwald et al. [51] proposed a hot path prediction method and achieved a good accuracy rate, thus reducing the execution information that needs to be collected; Porto et al. [52] proposed an automaton to optimize trace execution; or the running efficiency can be improved by executing trace in parallel [53].
In addition, by merging regions that exit early [54], using information such as branch type and control flow analysis [55], using branch history information [56], and inlining technology based on indirect jumps [57] to construct or Optimize the generated trace or region.

The code cache stores translated basic block codes for future reuse.
According to the principle of locality, code cache also has a certain impact on performance, so there have been some studies on the storage and replacement strategies of basic blocks in code cache[58][59][60].
It is hoped that by optimizing the structure of the cache Or the arrangement of the code in the cache to improve the execution efficiency of the program.

\subsection{Instruction Level Optimization}
Due to the rapid development of SIMD extensions of different ISAs in different trends in recent years, such as ARM's NEON, X86's AVX256, AVX512, etc., between different ISAs or even between different versions of the same ISA, SIMD instructions are very different.
The difference, which makes the translation of SIMD instructions more difficult than ordinary instructions.
Fu et al. [37] proposed a SIMD dynamic translation framework, applied it to QEMU [1], and obtained a performance improvement of 1.84 to 6.81 times compared with the original.
There are also researchers who use general-purpose SIMD IR to reduce the difficulty of SIMD conversion [35][36]; or vectorize the old version of the legacy binary program so that it can benefit from the new version of the SIMD extension [38 ]; and the translation of asymmetric SIMD instructions [39] and the translation of fixed-vector-length to indefinite-vector-length SIMD instructions [40] have been studied.
Another part of the research focuses on SIMD register type mismatch [41], register mapping from ARM to X86 [42][43], translation of non-sequential SIMD memory access instructions and register mapping [44] and other register mapping related issues.
Wu et al. [45] proposed that SIMD resources can be fully utilized to improve DBT performance, especially SIMD registers can be used to make up for the shortage of host general-purpose registers in the register map.

\subsection{Hardware Accelerate}
Some researchers hope to design a more efficient DBT tool through the method of software and hardware collaboration.
In addition to accelerating memory address mapping and assisting in code cache management through hardware mentioned above, hardware features of MIPS, VLIW and other architectures can also be used to reduce translation overhead or speed up the translation process.
Translating X86 to MIPS XBAR[85] can directly or partially use the host machine's decoder (decoder) to decode, or use the floating-point arithmetic unit and floating-point registers to accelerate the operation of X86 stack floating-point instructions.
The system-level translation tool [87] studied by Rokicki et al. can translate MIPS instructions to VLIW.
After generating the IR, the translation tool will analyze the parallelism between instructions and call the scheduler of the VLIW hardware to adjust the order of the translated code, so that instructions that can be parallelized are placed in the same bundle.
Their subsequent research [88] added three new hardware on the original basis (the client ISA was changed from MIPS to RISC-V), so that the construction of IR, instruction scheduling and code generation were all handed over to specialized hardware to execute.


\section{Applications of DBT}


\section{Challenges and Future Directions}
\label{sec:comments}


\section{Conclusions}
\label{sec:conclusions}

Dynamic binary translation is an area that has been studied for several decades.
It arose from the requirements of binary compatibility and migration and was gradually used in binary analysis, binary security, virtualization, and simulation.
We reviewed existing studies in this area, focusing on various optimizations of DBT, categorizing and illustrating them.
We also introduce various applications of DBT.
At last, we listed the problems that still exist in DBT and analyzed the future development directions.
Given that performance will still be the most important issue in DBT, we also propose a possible methodology for DBT optimization.


\bibliographystyle{ACM-Reference-Format}
\bibliography{\jobname}

\end{document}
\endinput
%%
%% End of file `sample-acmsmall.tex'.
